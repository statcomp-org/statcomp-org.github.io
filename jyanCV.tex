\documentclass[Statistics]{vita}


% \usepackage{natbib}

\usepackage{url}
\usepackage{times}
\usepackage[hmargin={0.5in, 0.5in}, vmargin={0.4in, 0.4in}, includeheadfoot]{geometry}
\usepackage{hyperref}
\usepackage{enumitem}
\setlist[enumerate]{leftmargin=3mm}

\renewcommand{\familydefault}{\sfdefault}
% \renewcommand{\baselinestretch}{0.9} % reduce line spacing

\newcommand{\R}{R}

\begin{document}
\title{Jun ~ Yan}
\businessAddress{Department of Statistics\\
  University of Connecticut\\
  215 Glenbrook Rd. U-4120\\
  Storrs, CT 06269% \\
  % \href{https://www.linkedin.com/in/jyanstat/}{LinkedIn}
}
\homeAddress{Phone: (860) 486-3416\\
  Fax: (860) 486-4113\\
  Email: jun.yan@uconn.edu\\
  Web: \url{https://statcomp.org/}% \\
  % \href{https://scholar.google.com/citations?user=4jVhnnEAAAAJ&hl=en}{Google Scholar}
}

\begin{vita}
  \begin{Degrees}\label{degrees}
  \item Ph.D., 2003, Statistics, University of Wisconsin, Madison, WI
  \item M.A., 1998, Economics, University of Miami, Miami, FL
  \item M.Econ., 1996, Statistics, Renmin University of China, Beijing, China
  \item B.Econ., 1993, Statistics, Renmin University of China, Beijing, China
  \end{Degrees}
  \begin{ResearchInterests}
  \item Social Network, Spatial Extremes, Dynamic Survival Models, Multivariate Dependence, Clustered Data Analysis
  \item Statistical Computing, Biostatistics, Environmental Applications, Public Health Applications, Sports Analytics
  \end{ResearchInterests}
  \begin{Positions}
  \item Professor, Department of Statistics, University of Connecticut, 2015--Present.
  \item Affiliated Faculty, The Connecticut Institute for the Brain and Cognitive Sciences, 2023--Present.
  \item Research Fellow, Center for Population Health, University of Connecticut Health Center, 2007--Present.
  \item Associate Professor, Department of Statistics, University of Connecticut, 2010--2015.
  \item Affiliated Faculty, Center for Environmental Sciences and Engineering, University of Connecticut, 2007--Present.
  \item Assistant Professor, Department of Statistics, University of Connecticut, 2007--2010.
  \item Assistant Professor, Department of Statistics and Actuarial Science, The University of Iowa, 2003--2007.
  \item Statistical Consultant, CALS Statistical Consulting Lab, University of Wisconsin--Madison,  2001--2003. % July 2001 --- July 2003.
  \end{Positions}
  \begin{HonorsAndAwards}
  \item Elected Member, 2023, Connecticut Academy of Science and Engineering
  \item Fellow, 2022, Institute of Mathematical Statistics
  \item Fellow, 2017, American Statistical Association
  \item Elected Member, 2014, International Statistical Institute
  \item Outstanding Service Award, 2014, International Chinese Statistical Association
  \end{HonorsAndAwards}
  \begin{ProfessionalMemberships}
  \item American Association for the Advancement of Science (AAAS)
  \item American Geophysical Union (AGU) 
  \item American Statistical Association (ASA)
  \item International Chinese Statistical Association (ICSA)
  \item Institute of Mathematical Statistics (IMS)
  \end{ProfessionalMemberships}
  %%%%%%%%%%%%%%%%%%%%%%%%%%%% 
  \begin{Services}
  \begin{Editorial}
  \item 2020--present: Editor-in-Chief \emph{Journal of Data Science}.
  \item 2020--2021: Associate editor, \emph{Brazilian Journal of Probability and Statistics}.
  \item 2020: Co-Guest Editor, Special Issue on ``Data Science in Action in Response to the Outbreak of COVID-19'', Journal of Data Science.
  \item 2014--2020: Associate editor,  \emph{Ecological and Environmental Statistics}.
  \item 2014: Associate editor, \emph{STEM Forums in Statistics}.
  \end{Editorial}
  \begin{SocietyService}
  \item 2025--2027: Chair-Elect/Chair/Past-Chair of the ASA Section on Statistical Learning and Data Science.
  \item 2024--2025: ASA Representative, ACM ASA MAA SIAM Joint Taskforce on Undergraduate Data Science Competencies.
  \item 2022--2024: Member, ASA Committee on Data Science and AI
  \item 2021--2023: Chair-Elect/Chair/Past-Chair of the ASA Section on Statistical Computing.
  \item 2020: Co-Chair, ASA--Journal of Data Science webinar series on ``Data Science in Action in Response to the Outbreak of COVID-19'', April 17 --- July 24, 2020.
  \item 2018--2021: Member, IMS Committee to Select Editors.
  \item 2017--2020: Awards Chair, ASA Section on Statistical Computing and Section on Statistical Graphics.
  \item 2016--2020, Editor, LIDA Newsletter, The official newsletter of the Lifetime Data Analysis Section, American Statistical Association.
%  \item 2016--present, Department representative at the Caucus of Academic Reps, American Statistical Association.
  \item 2015--2016: Chair, IT Committee, International Chinese Statistical Association.
  \item 2012--2014: Editor-in-Chief, International Chinese Statistical Association Bulletin.
  \end{SocietyService}
  \begin{Conferences}
  \item 2024: Co-Chair, International Forum on Data Science 2024.
  \item 2024: Lead Organizer, Banff International Research Station Workshop 24w5284: \href{http://www.birs.ca/event/24w5284}{Statistical Aspects of Trustworthy Machine Learning} (with Stephanie Hicks, Keegan Korthauer, Xiaotong Shen, and Helen Zhang)
  \item 2019--present: Chair, Organizing Committee, Annual UConn Sports Analytics Symposium.
  \item 2022: Chair, Organizing Committee, Excellence in Statistical Science: Celebrating the 60th Anniversary of UConn Statistics.
  \item 2022: Chair, Steering Committee, ASA Section on Statistical Computing Mini-Symposium: Statistical Computing in Action.
  \item 2022: Program Committee, 2023 ICSA Applied Statistics Symposium.
  \item 2020: Chair, Organizing Committee, The 3rd International Forum on Statistical Science and Big Data, June 13--15, 2020, School of Statistics, Shanxi University of Finance and Economics.
  \item 2019: Chair, Organizing Committee, 33rd New England Statistics Symposium: Statistical Data Science in Action.
  \item 2018: Organizing Committee, The 8th International Statistics Forum, Renmin University of China.
  \item 2018: Co-Chair, Short-course Committee, 2018 ICSA Applied Statistics Symposium.
  \item 2017: Co-Chair, Local Organizing Committee,  Conference on Lifetime Data Science: Data Science, Precision Medicine and Risk Analysis with Lifetime Data at University of Connecticut, May 25--27, 2017, Sponsored by the Lifetime Data Analysis Interest Group, American Statistical Association.
  \item 2015: Chair, Poster Committee, 2015 ICSA Applied Statistics Symposium
  \end{Conferences}
  \begin{GrantReviews}
  \item NSF CAIG panel review, 2024.
  \item NSF DMS CDSE-MSS panel review, 2016.
  \item NSF DMS Statistics panel review, 2013.
  \end{GrantReviews}
  \begin{JournalReviews}
  \item Annals of Applied Statistics; Annals of Statistics; Applied Stochastic Models in Business and Industry; Bernoulli; Biometrics; Biometrika; Bioscience; Canadian Journal of Statistics; Computational Statistics; Computational Statistics and Data Analysis; Econometrics and Statistics; Environmental and Ecological Statistics; IEEE Transactions on Neural Networks; International Journal of Forecasting; Journal of the American Statistical Association ;Journal of Data Science; Journal of Multivariate Statistics; Journal of Royal Statistical Society Series B; Journal of Royal Statistical Society Series C; Journal of Statistical Software; R Journal; Statistics and Probability Letters; Statistica Sinica; Statistics in Medicine; Stochastic Environmental Research and Risk Assessment; among others.
  \item Statistical Reviewer:  Journal of Applied Physiology; Advances in Physiology Education.
  \end{JournalReviews}
  \begin{DepartmentalService}
  \item 2023--2024: Promotion, Tenure, and Reappointment Committee (chair); X + 3 + 1 Admission (chair); Search Committee for Assistant Professors; Computer Committee; Graduate Examination Committee.
  \item 2022--2023: Promotion, Tenure, and Reappointment Committee (chair); 10-Year Strategic Plan Committee (chair); Organizing Committee of the 60th Anniversary Celebration (chair); X + 3 + 1 Admission (chair); Search Committee for Assistant Professors; Computer Committee; Graduate Examination Committee; SET+ Faculty Teaching Evaluations.
  \item 2021--2022: Promotion, Tenure, and Reappointment Committee (chair); 10-Year Strategic Plan Committee (chair); Organizing Committee of the 60th Anniversary Celebration (chair); X + 3 + 1 Admission (chair); Search Committee for Assistant Professor in Data Science; Computer Committee; Graduate Examination Committee. 
  \item 2020--2021: Data Science Program (chair), Computer Committee (chair), Graduate Examination,  X + 3 + 1 Admission (Chair), Promotion, Tenure, and Reappointment Committee.
  \item 2019--2020: Data Science Program (chair), Computer Committee (chair), Graduate Examination,  X + 3 + 1 Admission, Promotion, Tenure, and Reappointment Committee.
  \item 2018--2019: Search Committee (chair), Data Science Program (chair), Computer Committee (chair), Graduate Examination, Distinguished Statistician Series, 3 + 1 Admission, Promotion, Tenure, and Reappointment Committee.
  \item 2017--2018: Data Science Program (chair), Computer Committee (chair), Graduate Examination, Distinguished Statistician Series, 3 + 1 Admission, Graduate Admission Committee, Promotion, Tenure, and Reappointment Committee.
  \item 2016--2017: Computer Committee (chair), Program Review Committee, Graduate Students and Distinguished Alumni Awards Committee, X + 3 + 1 Admission, New England Statistics Symposium (chair), New England Statistics Society, Faculty Search Committee, Graduate Admission Committee, Promotion, Tenure, and Reappointment Committee.
  \item 2015--2016: Computer Committee (chair), Admission Committee, Department Head Search Committee, Promotion, Tenure, and Reappointment Committee.
  \item 2014--2015: Computer Committee (chair), Admission Committee, New England Statistics Symposium Committee.
  \item 2013--2014: Computer Committee (chair), Admission Committee.
  \item 2012--2013: Computer Committee (chair), Associate/Full professor search committee, Department's 50th Year Celebration committee, New England Statistics Symposium Committee, Admission Committee.
  \item 2011--2012: Computer Committee (chair), Admissions Committee, Biostatistics Search Committee,  Department's 50th Year Celebration committee, Biostatistics Program Development Committee.
  \item 2010--2011: Colloquium Committee (chair), Library/Tech Reports Committee, Search Committee, Biostatistics Program Development Committee, New England Statistics Symposium Committee.
  \item 2008--2010: Library/Tech Reports Committee, Biostatistics Program Development Committee, Social Committee.
  \item 2006--2007: Colloquium Committee (chair), M.S. Exam -- Minor Committee (chair), Computer Committee.
  \item 2005--2006: Colloquium Committee, M.S. Exam -- Minor Committee, Search Committee, and Social Committee.
  \item 2004--2005: M.S. Exam -- Minor Committee, Search Committee, and Social Committee (chair).
  \item 2003--2004: Computer Committee, M.S. Exam -- Minor Committee, and Social Committee.
  \end{DepartmentalService}
  \begin{Outreaches}
  \item 2023--present: New York City Open Data Ambassador, working with libraries and community organizations to help bridging data literacy gaps and promoting neighborhood and issue-based dialogue.
  \item 2022--present: Instructor of Introduction to Data Science, one-week intensive Pre-College Summer course for high schoolers.
  \item 2019--2019: Instructor of Coding for Kids (6--8 years old), Connecticut  Chinese Language Academy.
  \end{Outreaches}
  \begin{Media}
  \item UConn Today (2024): \href{https://today.uconn.edu/2024/04/uconn-sports-analytics-symposium-returns-for-fifth-time/}{UConn Sports Analytics Symposium Showcases the Numbers Behind the Games}.
  \item UConn Today (2023): \href{https://today.uconn.edu/2023/03/uconn-students-to-showcase-work-at-nyc-open-data-week/}{UConn Students to Showcase Work at NYC Open Data Week}.
  \item ASA Member News (2023): \href{https://www.amstat.org/docs/default-source/amstat-documents/data-is-my-job-poster.pdf}{Data is My Job}.
  \item AMStat News (2023): \href{https://magazine.amstat.org/blog/2023/02/01/uconn-stats-dept-anniversary/}{UConn Department of Statistics Celebrates 60th Anniversary}.
  \item AMStat News (2023): \href{https://magazine.amstat.org/blog/2023/01/02/uconn-sports-analytics-symposium/}{UConn Sports Analytics Symposium Boasts New Features}.
  \item UConn Today (2022): \href{https://today.uconn.edu/2022/10/from-wyoming-mountains-to-connecticut-forests-tracking-feline-apex-predators/}{From Wyoming Mountains to Connecticut Forests, Tracking Feline Apex Predators}.
  \item Methods Blog (Methods in Ecology and Evolution) (2022): \href{https://methodsblog.com/2022/01/21/revealing-the-hidden-lives-of-cryptic-mountain-lions-using-gps-data-and-a-moving-resting-motion-model/}{Revealing the hidden lives of cryptic mountain lions using GPS data and a Moving-Resting Motion model}.
  \item AMStat News (2022): \href{https://magazine.amstat.org/blog/2022/01/01/uconn-sports-analytics/}{UConn Sports Analytics Symposium a Home Run}.
  \item UConn Today (2020): \href{https://today.uconn.edu/2020/10/uconn-holds-second-annual-sports-analytics-conference/}{UConn Holds Second Annual Sports Analytics Conference}.
  \item UConn Today (2020):
\href{https://today.uconn.edu/2020/04/clas-faculty-students-shifting-work-covid-19/}{CLAS Faculty and Students Shifting Work to COVID-19}.
  \item UConn Today (2019): \href{https://today.uconn.edu/school-stories/uconn-hosts-new-sports-analytics-symposium/}{UConn Hosts New Sports Analytics Symposium}.
  \end{Media}
\end{Services}
%%%%%%%%%%%%%%%%%%%%%%%%%%%%%%%
  \begin{TeachingExperience}
    \begin{UConn}
    \item Data Science: One-week intensive Pre-College Summer course for high schoolers; Summer 2021; Summer 2022; Summer 2023.
    \item Special topics: Advanced Data Manipulation and Analysis with Python (4185), 1-credit undergraduate level: Fall 2022; Spring 2023; Fall 2023.
    \item Undergraduate Seminar/Investigation of Special Topics: Statistical Writing (3494W/5095): 1-credit graduate/undergraduate statistical literacy requirement; Fall 2022.
    \item Introduction to Data Science (5255/3255): 3-credit graduate/undergraduate level; Fall 2021; Spring 2022; Spring 2023; Spring 2024.
    \item Biostatistics (5625/4625), 3-credit graduate/undergraduate level; Spring 2021.
    \item Data Science in Action (6494), 3-credit seminar; Spring 2018 (co-taught with Kun Chen and Elizabeth Schifano); Spring 2019.
    \item Statistical Computing (5361), 3-credit graduate level; Spring 2018; Fall 2018; Fall 2020.
    \item Applied Statistics II (5605), 3-credit graduate level (required for qualifying exam); Spring 2017.
    \item Multivariate Modeling with Copulas (6494), 3-credit graduate level; Fall 2016.
    \item Advanced Statistical Computing (6494), 3-credit graduate level; Spring 2014/2015/2016.
    \item Applied Longitudinal Data Analysis (6494), 3-credit graduate level; Fall 2014.
    \item Mathematical Statistics I/II (5585/5685), 3-credit graduate level sequence (required for qualifying exam); Fall 2011 --- Spring 2013, Fall 2007 --- Spring 2010, Fall 2015, Fall 2017.
    \item Data Analysis Using R (6494), 3-credit graduate level; Spring 2011.
    \item Environmental Statistics (6494), 3-credit graduate level; Fall 2010.
    \item Statistical Methods (220), 3-credit undergraduate level; Spring 2008.
    \end{UConn}
    \begin{UIowa}
    \item Probability and Stochastic Processes I (22s:195), Fall 2006.
    \item Mathematical Statistics II (22s:154), Spring 2007, Spring 2006, Spring 2004.
    \item Mathematical Statistics I (22s:153), Fall 2005, Fall 2004, Fall 2003.
    \item Applied Time Series Analysis (22s:156), Spring 2007, Spring 2005.
    \end{UIowa}
  \end{TeachingExperience}
  \begin{Students}
    \begin{Ph.D.}
    \item Qingkai Dong, Ph.D. expected 2028 (joint with HaiYing Wang): Subsampling and adaptive design.
    \item Shiying Xiao, Ph.D. expected 2027: Network analytics.
    \item Zefang Min, Ph.D. expected 2026: Time series and causal inference.
    \item Xiaomin Lu, Ph.D. expected 2025: Survival analysis and causal inference.
    \item Sydney Louit, Ph.D. expected 2025: Network analytics.
    \item Zhenyu Xu, Ph.D. expected 2025: Cure rate and competing risk.
    \item Jun Bruce Jin, Ph.D. 2024 (joint with Kun Chen): On large-scale transfer learning with heterogeneous data. Placement: Faculty Biostatistician, Henry Ford Health.
    \item Surya Eada, Ph.D. 2024 (joint with Vladimir Pozdnyakov): L\'evy process governed by telegraph signal process: Statistical Inferences and applications. Placement: Assistant Professor of Teaching, Oregon State University.
    \item Yingfa Xie, Ph.D. 2024: Recurrent events modeling based on a reflected Brownian motion with application to hypoglycemia. Placement: Postdoc researcher, Yale University.
    \item Lucas Godoy, Ph.D. 2024: Hausdorff-Gaussian process with spatial and spatiotemporal applications. Placement: Postdoc researcher, University of California at Santa Cruz.
    \item Yelie Yuan, Ph.D. 2023: On assortativity of weighted directed networks. Placement: Consumer and Community Banking Risk Program Associate,  JP Morgan.
    \item Zehan Yang, Ph.D. 2023 (joint with HaiYing Wang): Optimal subsampling methods for massive survival data using accelerated failure time models. Placement: Mathematical Statistician, US Food and Drug Administration.
    \item Jackson Lautier, Ph.D. 2023 (joint with Vladimir Pozdnyakov): Essays on discrete-time survival analysis with applications to securitization and consumer finance. Placement: Assistant Professor, Bentley University. 
    \item Sai Ma, Ph.D. 2022: Optimal fingerprinting with estimating equations. Placement: Statistician, Vertex Pharmaceuticals.
    \item Abby Lau, Ph.D. 2022: Extreme value modeling with errors-in-variables in detection and attribution of changes in climate extremes. Placement: Postdoc researcher, University of Pennsylvania.
    \item Yan Li, Ph.D. 2021 (joint with Kun Chen): Amalgamation-based statistical learning for compositional data. Placement: Postdoc researcher, University of Michigan.
    \item Jieying Jiao, Ph.D. 2020: On Bayesian methods for spatial point processes. Placement: Senior Consultant, Travelers Insurance.
    \item Chaoran Hu, Ph.D. 2020 (joint with Vladimir Pozdnyakov): On Brownian motion governed by telegraph process. Placement: Research Scientist, Ely Lilly and Company.
    \item Wenjie Wang, Ph.D. 2019 (joint with Kun Chen): Integrated survival analysis with application to suicide risk. Placement: Research Scientist, Eli Lilly and Company.
    \item Yishu Xue, Ph.D. 2019 (joint with Elizabeth Schifano): Diagnostic methods for big survival data. Placement: Senior Consultant, Travelers Insurance.
    \item Greg Vaughan, Ph.D. 2017 (joint with Kun Chen): Stagewise generalized estimating equations. Placement: Assistant Professor, Bentley University.
    \item Yujing Jiang, Ph.D. 2017: Marginal score equations for spatial extremes with latent signals and applications to fingerprinting changes in climate extremes. Placement: Postdoc researcher, Colorado State University.
    \item Brian Bader, Ph.D. 2016: Automated, efficient, and practical extreme value analysis with environmental applications. Placement: Statistician, KPMG.
    \item Chun Wang, Ph.D. 2016 (joint with Elizabeth Schifano): Online updating methods for big data streams. Placement: Senior Analyst, Liberty Mutual Insurance.
    \item Zhuo Wang, Ph.D. 2015: Estimating equations for spatial extremes with application to detection and attribution analysis of changes in climate extremes. Placement: Assistant Professor, Shenzhen University, China.
    \item Hongwei Shang: Ph.D. 2013: A Two-step estimation procedure and a goodness-of-fit test for spatial extremes models. Placement: Statistician, HP Analytic Lab.
    \item Sy Han (Steven) Chiou, Ph.D. 2013 (joint with Sangwook Kang): Statistical methods and computing for semiparametric accelerated failure time model with induced smoothing. Placement: Assistant Professor, Department of Mathematics and Statistics, University of Minnesota Duluth.
    \item Xiaojing Wang: Ph.D. 2011: Statistical inferences for interval censored data. Placement: Quantitative Analyst, Google New York.
    \item Marcos Prates, Ph.D. 2011 (joint with Dipak Dey): Link specification and spatial dependence for generalized linear mixed models. Placement: Assistant Professor, Departamento de Estat\'{i}stica, Universidade Federal de Minas Gerais, Brazil.
    \end{Ph.D.}
    \begin{UndergraduateHonor}
    \item Mathew Chandy (2024): Nonparametric block bootstrap Kolmogorov--Smirnov goodness-of-fit test. Placement: Ph.D. Student in Statistics, UCLA.
    \item Kathleen Houlihan (2024): Selecting team members for the female Artistic Gymnastics Team USA for the Paris Olympics. Placement: Dental School student, Boston University.
    \item Shannon Yeung (2023): Varying effects of short term interest rates. Placement: GMI Analyst, BNY Mellon.
    \item Owen Fiore (2023): Was Devon Allen unjustly disqualified at the 2022 World Track and Field Championships? Placement: MS Student in Data Science, University of Connecticut.
    \item Pranav Tavildar (2023): Sentiment analysis of twitter in relation to fossil fuel stock Prices. Placement: MS Student in Data Analytics, Georgia Institute of Technology.
    \item Samuel Hughes (2022): Statistical evaluation of field hockey penalty corners. Placement: MS Student in Data Science, Northeastern University.
    \item Anthony Zeimbekakis (2022): On misuses of the Kolmogorov-–Smirnov test for one-sample goodness-of-fit. Placement: Analyst Development Program, Electric Insurance Company.
    \item Brian Krikorian (2022): Points above replacement: A new NBA metric to evaluate player performance. Placement: Success Metrics Intern, UMass Chan Medical School.
    \item Justin Franklin (2021): Comparison of fraud detection methods: A case study with insurance. Placement: Software Engineer, Travelers.
    \item Andrew Tammaro (2021): NFL front office analytics with R. Placement: Corporate Strategy and Data Analytics Intern, 1BusinessWorld.
    \item Michael Price (2021): The effects of the NBA COVID bubble on the NBA playoffs: A case study for home-court advantage. Placement: MS program in Applied Statistics, University of Delaware.
    \item Dylan Barrett (2020): When is the best time to steal bases? Placement: Customer Operations Agent, FanDuel.
    \item Taaj Cheema (2020): An analysis of Oliver’s four factors in the golden age of NBA offense. Placement: Data Scientist, IBM.
    \item Jack Schooley (2020): Predicting the outcomes of soccer games. Placement: MS Student in Data Sience, MIT.
    \item Thomas Kennon (2018): Finding an ultimate limit for an NBA player's shooting percentage. Placement: Data Engineer, The Hartford.
    \item Junghi Kim (2010, Joint with Evarist Gin\'e and Nalini Ravishanker): A close look at marginal Cox model and conditional Cox model with application of recurrent gap times. Placement: Ph.D. Student in Biostatistics, University of Minnesota.
    \end{UndergraduateHonor}
  \end{Students}
  \begin{TeachingAccomplishments}
    \begin{StudentAwards}
    \item Jackson Lautier (2024): Honorable Mention, Arnold Zellner Thesis Award in Econometrics and Statistics, American Statistical Association.
    \item Sydney Louit (2024): Honorable Mention, Student Paper Award, 2024 Applied Statistics Symposium, International Statistical Association.
    \item Lucas Godoy (2024): Honorable Mention, Student Paper Award, Section on Statistics and the Environment, American Statistical Association.
    \item Yingfa Xie (2024): Student Paper Award, Section on Lifetime Data Science, American Statistical Association.
    \item Zehan Yang (2023): Honorable Mention, Student Paper Award, 2023 Applied Statistics Symposium, International Chinese Statistical Association.
    \item Yelie Yuan (2023): Honorable Mention, John M. Chambers Statistical Software Award, American Statistical Association.
    \item Jackson Lautier (2022): Student Paper Award, Special Conference Celebrating the 60th Anniversary of UConn Department of Statistics.
    \item Zehan Yang (2022): Student Paper Award, Section on Lifetime Data Science, American Statistical Association.
    \item Jackson Lautier (2022): Student Paper Award, Risk Analysis Section, American Statistical Association.
    \item Jun Jin (2022): Honorable Mention, Student Paper Award, Risk Analysis Section, American Statistical Association.
    \item Chaoran Hu (2020): Student Paper Award, Statistical Computing and Statistical Graphics Sections, American Statistical Association.
    \item Yan Li (2020): ENAR Distinguished Student Paper Award, International Biometric Society.
    \item Yan Li (2019): IBM Student Paper Award, 33rd New England Statistics Symposium.
    \item Yishu Xue (2019): ENAR Distinguished Student Paper Award, International Biometric Society.
    \item Wenjie Wang (2017): IBM Student Paper Award, 31st New England Statistics Symposium.
    \item Gregory Vaughan (2017): Student Paper Award, Mental Health Section, American Statistical Association.
    \item Steven Chiou (2012): Student Paper Award, Applied Statistics Symposium, International Chinese Statistical Association.
    \item Yung-wei Chen (2012): Student Paper Award,  Social Statistics, Government Statistics, and Survey Research Methods Sections,  American Statistical Association.
    \item Xiaojing Wang (2010): IBM Student Paper Award, 24th New England Statistics Symposium.
    \end{StudentAwards}
  %   \item Cultivated interdisciplinary collaboration on species diversity in Marcos Prates' thesis research.
  %   \item Cultivated interdisciplinary collaboration on hydrological engineering in Xiaojing Wang's research.
  %   \item Cultivated interdisciplinary collaboration on climate change in Hongwei Shang's thesis research.
  %   \item Cultivated interdisciplinary collaboration on adolescent depression in Steven Chiou's thesis research.
  %   \item Cultivated interdisciplinary collaboration on animal movement and on nonprofit organization study in Yung-wei Chen's research.
  %   \item Coauthored accepted journal papers and submitted manuscripts with every graduate student advisee.
    \begin{TeachingHighlights}
    \item Undergraduate advisees have published journal articles since 2022.
    \item Advised the Undergraduate Data Science Club as faculty advisor since 2019.
    \item Co-taught ``Data Science'' to pre-college students in the UConn Pre-College Summer Program since 2020.
    \item Developed new courses for the data science program: Statistical Data Science in Action; Introduction to Data Science; Spatiotemporal Statistics.
    \item Advised undergraduates in data science in UConn Individualized Major Program.
    \item Mentored statistics students to develop software packages in R.
    \item Taught students in ecology and evolutionary biology how to do data analysis with R.
    \item Taught the first-year graduate required sequence of mathematical statistics effectively.
    \end{TeachingHighlights}
  %   \item Have been maintaining R and R packages on the beowulf cluster heavily used by students doing research.
  %   \item Helped with all computing related problems from all students in the department.
  \end{TeachingAccomplishments}
  % \begin{CollaborativeWork}
%   %\item Inter-calibration of measures from different remote sensor systems --- with a geographer at the University of Iowa
%   \item Testing concordance in clinical traits in familial studies of Inflammatory Bowel Disease --- with a gastroenterologist at the University of Iowa
%   \item Parallelizing Bayesian analysis for geostatistical models with large environmental data --- with a geographer and an information scientist at the University of Iowa
%   \item Comparison of fungi species abundance distributions --- with a biologist at University of Wisconsin
%   \item Computing in optimal investment problems --- with an actuary at the University of Iowa
%   \item A survey of spatial econometrics --- with econometricians at University of Illinois -- Urbana-Champaign
%   \end{CollaborativeWork}
  \begin{Grants}
    \begin{External}
    \item Connecticut Children's Medical Center, 04/01/2024---03/31/2025: Predictive and Analytical Tools for Decision Making at Connecticut Children’s Medical Center. \$23,228.79. PI: Jun Yan.
    \item Servier, 03/11/2024---12/31/2024: AI-Based Adaptive Clinical Trail. \$21,666.67.  PI: Jun Yan.
    \item NSF DMS2219336, 09/01/2022 --- 08/31/2025: Conference: UConn Sports Analytics Symposium: Engaging Students into Data Science. \$49,986. PI: Jun Yan; Co-PIs: Laura Burton, Kun Chen,  Robert Huggins, Elizabeth Schifano.
    \item NSF DMS2210735: 08/01/2022 --- 07/31/2025: Models and Inferences for Heterogeneous Interaction Patterns in Social Networks. \$360,000. PI: Jun Yan; Co-PI: Xianyang Zhang.
    \item NSF CC19325716, 08/01/2019 --- 07/31/2021: CC* Compute: Shared Computing Infrastructure for Large-scale Science Problems. \$400,000. PI:  Richard T. Jones, Co-PIs: Vernon Cormier, Kyungseon Joo, Cara D. Battersby, and Jun Yan.
    \item NSF DMS1521730, 2015/09/01 --- 2018/8/31: Fingerprint Methods for Detection and Attribution of Changes in Climate Extremes with Spatial Estimating Equations. \$100,000. PI: Jun Yan.
    \item NSF DMS1209022, 08/15/2012 --- 07/31/2015: Statistical Inferences, Computing, and Applications for Semiparametric Accelerated Failure Time Models. \$130,000. PI: Jun Yan. Co-PI: Sangwook Kang.
    \item University of Wisconsin (NIH RO1 subcontract, PI: Hui-Chuan Lai), 09/01/2011 --- 08/31/2016: Newborn Screening, Malnutrition and Lung Disease in Children with Cystiv Fibrosis. \$37,284. PI: Jun Yan.
    \item NASA NNX10AG77G, 06/01/2010 --- 05/31/2013: Testing the Suitability of Satellite Precipitation Products for Hydrological Modeling at Multiple Scales across the Blue Nile Basin. \$308,845. PI: Mekonnen Gebremichael; Co-PI Jun Yan.
    \item NOAA NA10NWS4680004, 05/01/2010 --- 04/30/2012: National Weather Service (NWS) Collaborative Science Technology, and Applied Research (CSTAR) --- A New Statistical Model of Streamflow Forecast Error. \$148,574. PI: Mekonnen Gebremichael; Co-PI Jun Yan.
    \item NSF DMS0805965, 07/01/2008 --- 06/30/2011: Unified Dynamic Modeling of Event Times with Semiparametric Profile Estimating Functions: Theory, Computing, and Applications. \$150,000. PI: Jun Yan.
    \item NSF DMS0618883, 07/15/2006 --- 07/14/2007: Statistical Computing Research Environments (SCREMS). \$95,000. PI: Mary Kathryn Cowles; Co-PIs: John Geweke, Jian Huang, Luke Tierney, and Jun Yan.
%   \item 2006--2007: TeraGrid --- Parallelizing MCMC for Bayesian Estimation and Prediction in Spatiotemporal Geostatistical Models. PI: Jun Yan, Co-PIs: Marc P. Armstrong, Mary Kathryn Cowles, Brian J. Smith, and Shaowen Wang.
%   \item 2006--2007: TeraGrid --- Developing GISolve as a GIScience Gateway Toolkit for Geographic Information Analysis. PI: Shaowen Wang, Co-PIs: Mary Kathryn Cowles, Marc P. Armstrong, and Jun Yan.
%   \item 2004--2005: TeraGrid --- Spatial Statistics Middleware for Markov-chain-Monte-Carlo Bayesian Geostatistical Models. PI: Mary Kathryn Cowles, Co-PIs: Marc P. Armstrong, Shaowen Wang, and Jun Yan.
  \end{External}
  \begin{Internal}
  \item June 2024 -- May 2025, The Connecticut Institute for the Brain and Cognitive Science, University of Connecticut ---  Identifying the association between longitudinal changes in functional connectivity and Alzheimer's disease progression.
  \item January 2017 -- December 2019, Innovative Education in Science, College of Liberal Arts and Sciences, University of Connecticut --- Data Science Lab: Real World Data Science Problems Meet Future Data Scientists (with Kun Chen and Elizabeth Schifano).
  \item April 2015 -- March 2016, Research Excellence Program, University of Connecticut --- Statistical Methods and Computing for Detection and Attribution of Changes in Climate Extremes.
  \item January 2011 -- May 2011, Multidisciplinary Environmental Research Award, Center for Environmental Sciences and Engineering, University of Connecticut --- A Constrained Stochastic Model for Animal Movement Data with Application to Deer Home Range (with Thomas Meyer).
  \item January 2010 -- December 2010, Faculty Large Grant, University of Connecticut --- Semiparametric Methods for Spatial Extremes with Application to Extremal Peak Flow in Connecticut.
  \item January 2008 -- December 2008, Faculty Large Grant, University of Connecticut --- Partly Functional Temporal Process Regression with Semiparametric Profile Estimating Functions: Theory and Application.
  \item January 2008 -- May 2008, Multidisciplinary Environmental Research Award, Center for Environmental Sciences and Engineering, University of Connecticut --- A Hierarchical Spatio-Temporal Model for Terrestrial Snails Abundances in a Tropical Forest (with Michael Willig).
  \item July 2006 -- June 2007, Mathematical \& Physical Sciences Funding Program (MPSFP), University of Iowa --- Partly Functional Temporal Process Regression.
  \item January 2004 -- December 2004, Mathematical \& Physical Sciences Funding Program (MPSFP), University of Iowa --- Nonparametric Inference for Nonstationary Stochastic Processes.
  \end{Internal}
\end{Grants}
\begin{Publications}
%   \renewcommand{\refname}{\bf\sffamily Refereed Articles}
%   \input{jyanpub.bbl}
%     \renewcommand{\refname}{\bf Refereed Articles}
%      \bibliographystyle{asa} % Choose ASA style for bibliography
%      \bibliography{jyan}
%      \nocite{*}
  \begin{Books}
  \item Hofert, M., Kojadinovic, I., M\"achler, M., and Yan, J. (2018): {\em Elements of Copula Modeling with R\/}. Springer. 
  \item Dey, D.~K. and Yan, J. (eds.) (2015): {\em Extreme Value Modeling and Risk Analysis: Methods and Applications\/}. Chapman \& Hall/CRC.
  \end{Books}
  \begin{BookChapters}
  \item Yin, F., *Jiao, J., Yan, J., and Hu, G.. (2022): Bayesian nonparametric estimation for point processes with spatial homogeneity: A spatial analysis of NBA shot locations. {\em Proceedings of the 39th International Conference on Machine Learning}. 162: 25523--25551.
  \item *Vaughan, G., Aseltine, R., Chiou, S., and Yan, J. (2016): An alarm system for flu outbreaks using Google Flu Trend data. In J.~Lin, B.~Wang, X.~Hu, K.~Chen, and R.~Liu (eds.) {\em Statistical Applications from Clinical Trials and Personalized Medicine to Finance and Business Analytics}, pp.293--304, Springer.
  \item *Chiou, S., Kang, S., and Yan, J. (2015): Change point analysis of top baseball batting average. In D.~K.~Dey and J.~Yan (eds.) {\em Extreme Value Modeling and Risk Analysis: Methods and Applications\/}, pp.493--504, Chapman \& Hall/CRC.
  \item Dey, D.~K., Roy, D., and Yan, J. (2015): Univariate extreme value analysis. In D.~K. Dey and J.~Yan (eds.), {\em Extreme Value Modeling and Risk Analysis: Methods and Applications\/}, pp.1--22, Chapman \& Hall/CRC.
  \item *Jiang, Y., Dey, D.~K., and Yan, J. (2015): Multivariate extreme value analysis. In D.~K. Dey and J.~Yan (eds.), {\em Extreme Value Modeling and Risk Analysis: Methods and Applications\/}, pp.23--39, Chapman \& Hall/CRC.
  \item *Wang, X., Sinha, A., Yan, J., and Chen, M.-H. (2012): Bayesian inference of interval-censored survival data. In D.-G. Chen, J.~Sun, and  K.~E. Peace (eds.), {\em Interval-Censored Time-to-Event Data: {M}ethods and Applications\/}, pp.167--196, Chapman \& Hall/CRC.
  \item Yan, J. (2006): Multivariate modeling with copulas and engineering applications. In H.~Pham (ed.), {\em Handbook of Engineering Statistics\/}, pp. 973--990, Springer.
  \end{BookChapters}
  \begin{RefereedJournalArticles}
  \item *Lautier, J., Pozdnyakov, V., and Yan, J. (2022): On the convergence of credit risk in current consumer automobile loans. {\em Journal of the Royal Statistical Society: Series A\/}. Forthcoming.
  \item *Xie, Y., Fu, H., Huang, Y., Pozdnyakov, V., and Yan, J. (2024+): Recurrent events modeling based on a reflected Brownian motion with application to hypoglycemia. {\em Biostatistics\/}. Forthcoming.
  \item *Xu, Z., Fine, J. P., Song, W., and Yan, J. (2024+): On GEE for mean-variance-correlation models: Variance estimation and model selection. {\em Statistics in Medicine\/}. Forthcoming.
  \item *Eada, S. T., Pozdnyakov, V., and Yan, J. (2024+): Discretely observed Brownian Motion governed by telegraph signal process: Estimation and applications to finance. {\em Statistical Inference for Stochastic Processes\/}. Forthcoming.
  \item *Jiao, J., Song, W., Xue, Y., and Yan, J. (2024+): Heteroscedastic growth curve modeling with shape-restricted splines. {\em New England Journal of Statistics in Data Science\/}. Forthcoming.
  \item *Lautier, J., Pozdnyakov, V., and Yan, J. (2023+): Estimating a discrete distribution subject to random left-truncation with an application to structured finance. {\em Econometrics and Statistics\/}. Forthcoming. 
  \item *Wang, W., Luo, C., Aseltine, R. H., Wang, F., Yan, J., and Chen, K. (2023+): Survival modeling of suicide risk with rare and uncertain diagnoses. {\em Statistics in Biosciences\/}. Forthcoming.
  \item Carter, E. J., *Lau, Y. T. A., Buchanan, L., Krol, D. M., Yan, J., and Aseltine, R. H. (2024): Accountable care organizations and {HPV} vaccine uptake: A multilevel analysis. {\em American Journal of Managed Care\/}. 30(10): e282--e288.
  \item *Chandy, M., Schifano, E. D., and Yan, J. (2024): On sample size needed for block bootstrap confidence intervals to have desired coverage rates. {\em American Journal of Undergraduate Research\/}. 20(4): 3--16.
  \item Choi, D., Bae, W., Yan, J., and Kang, S. (2024): A general model-checking procedure for semiparametric accelerated failure time models. {\em Statistics and Computing\/}. 34(3): 117.
  \item *Houlihan, K. and Yan, J. (2024): How to build a gymnastics team. {\em Significance\/}, 21(3), 10--14. 
  \item *Jin, J., Yan, J., Aseltine, R. H., and Chen, K. (2024) Transfer learning with quantile regression. {\em Technometrics\/}. 66(3): 381--393.
  \item *Lautier, J., Pozdnyakov, V., and Yan, J. (2024): On the maximum likelihood estimation of a discrete, finite support distribution under left-truncation and competing risks. {\em Statistics and Probability Letters\/}. 207: 109973.
  \item *Yang, Z., Wang, H., and Yan, J. (2024): Optimal subsampling for semi-parametric accelerated failure time models with massive survival data using a rank-based approach. {\em Statistics in Medicine\/}. 45(24): 4650--4666.
  \item *Yang, Z., Wang, H., and Yan, J. (2024): Subsampling approach for least squares fitting of semi-parametric accelerated failure time models to massive survival data. {\em Statistics and Computing\/}. 34(2): 77.
  \item *Yuan, Y., Yan, J., and Zhang, P. (2024): A strength and sparsity preserving algorithm for generating weighted, directed networks with predetermined assortativity. {\em Physica A: Statistical Mechanics and Its Applications\/}. 638: 129634.
  \item *Zeimbekakis, A., Schifano, E. D., and Yan, J. (2024): On misuses of the Kolmogorov--Smirnov test for one-sample goodness-of-fit. {\em American Statistician\/}. 78(4): 481--487.
  \item Bar, H. and Yan, J. (2023): Legendary career and colorful life: A conversation with Dr. Bob Riffenburgh. {\em Journal of Data Science\/}. 21(4): 818--837.
  \item Chiou, S., Xu, G., Yan, J., and Huang, C.-Y. (2023): Regression modeling for recurrent events possibly with an informative terminal event using R package {reReg}. {\em Journal of Statistical Software\/}. 105(1): 1--34.
  \item *Lau, A. Y. Z, Wang, T., Yan, J., and Zhang, X. (2023): Extreme value modeling with errors-in-variables in detection and attribution of changes in climate extremes. {\em Statistics and Computing\/}. 33(6): 125.
  \item *Lautier, J. P., Pozdnyakov, P. and Yan, J. (2023): Pricing time-to-event contingent cash flows: A discrete-time survival analysis approach. {\em Insurance: Mathematics and Economics\/}. 110: 53--71.
  \item *Li, Y., Chen, K., Yan, J., and Zhang, X. (2023): Regularized fingerprinting in detection and attribution of climate change with weight matrix optimizing the efficiency in scaling factor estimation. {\em Annals of Applied Statistics\/}. 17(1): 225--239.
   \item *Ma, S., Wang, T., Yan, J., and Zhang, X. (2023): Optimal fingerprinting with estimating equations. {\em Journal of Climate\/}. 36(20): 7109--7122.
  \item *Yuan, Y., Wang, T., Yan, J., and Zhang, P. (2023): Generating general preferential attachment networks with R package {wdnet}. {\em Journal of Data Science\/}. 21(3): 538--556.
  \item *Wang, F., Wang, H,, and Yan, J. (2023): Diagnostic tests for the necessity of weight in regression with survey data. {\em International Statistical Review\/}. 91(1): 55--71.
  \item *Hu, C., Pozdnyakov, V., and Yan, J. (2022): On occupation time for on-off processes with multiple off-states. {\em Modern Stochastics: Theory and Applications\/}. 9(4): 413--430.
  \item *Jiao, J., Tang, Z., Yue, M., Zhang, P., and Yan, J. (2022): Cyberattack-resilient load forecasting with adaptive robust regression. {\em International Journal of Forecasting\/}. 38(3): 910--919.
  \item *Lau, A. Y. Z and Yan, J. (2022): Bias analysis of generalized estimating equations under measurement error and practical bias correction. {\em Stat\/}. 11(1): e418.
  \item *Price, M. and Yan, J. (2022): The effects of the NBA COVID Bubble on the NBA playoffs: A case study for home-court advantage. {\em American Journal of Undergraduate Research\/}. 18(4): 3--15.
  \item Sun, Q., Zwiers, F., Zhang, X., and Yan, J. (2022): Quantifying the human influence on the intensity of extreme 1- and 5-day precipitation amounts at global, continental, and regional scales. {\em Journal of Climate\/}. 35(1): 195--210.
  \item Wang, T., Yan, J., *Yuan, Y., and Zhang, P. (2022): Generating directed networks with predetermined assortativity measures. {\em Statistics and Computing}. 32: 91.
  \item *Xiao, S., Yan, J., and Zhang, P. (2022): Incorporating auxiliary information in betweeness measure for input-output networks. {\em Physica A: Statistical Mechanics and Its Applications\/}. 607: 128200.
  \item *Yang, Z., Wang, H., and Yan, J. (2022): Optimal subsampling for parametric accelerated failure time models with massive survival data. {\em Statistics in Medicine}. 41(27): 5241--5431.
  \item Zhang, P., Wang, T., and Yan, J. (2022): PageRank centrality and algorithms for weighted, directed networks. {\em Physica A: Statistical Mechanics and Its Applications\/}. 586: 126438.
  \item Chang, S.-Y., Jin, J., Yan, J., Dong, X., Chaudhuri, B., Nagapudi, K, and Ma, A. W. K. (2021): Development of a pilot-scale HuskyJet binder jet 3D printer for additive manufacturing of pharmaceutical tablets. {\em International Journal of Pharmaceutics\/}. 605: 120791.
  \item Feng Chang, C., Garcia, V., Tang, C., Vlahos, P., Wanik, D., Yan, J., Bash, J. O., and Astitha, M. (2021): Linking multi-media modeling with machine learning to assess and predict lake chlorophyll-$\alpha$ concentrations. {\em Journal of Great Lakes Research\/}. 47(6): 1656--1670.
  \item *Hu, C., Elbroach, M., Meyer, T., Pozdnyakov, V., and Yan, J. (2021): Moving-resting process with measurement error in animal movement modeling. {\em Methods in Ecology and Evolution\/}. 12(11): 2221--2233.
  \item *Jiao, J., Hu, G., and Yan, J. (2021): A Bayesian marked spatial point processes for basketball shot chart. {\em Journal of Quantitative Analysis in Sports\/}. 17(2): 77--90. 
  \item *Jiao, J., Hu, G., and Yan, J. (2021): Heterogeneity pursuit for spatial point pattern with application to tree locations: A Bayesian semiparametric recourse. {\em Environmetrics\/}. 32(7): e2694.
  \item *Li, Y., Chen, K., Yan, J., and Zhang, X. (2021): Uncertainty in optimal fingerprinting is underestimated. {\em Environmental Research Letters\/}. 16(8): 084043.
  \item *Wang, T., Xiao, S., Yan, J., and Zhang, P. (2021): Regional and sectoral structures of the Chinese economy: A network perspective from multi-regional input-output tables. {\em Physica A: Statistical Mechanics and Its Applications\/}. 581: 126196.
  \item *Wang, W. and Yan, J. (2021): Shape-restricted regression splines with R package splines2. {\em Journal of Data Science\/}. 19(3): 498--517.
  \item *Wang, Z., Jiang, Y., Wan, H., Yan, J., and Zhang, X. (2021): Towards optimal fingerprinting in detection and attribution of changes in climate extremes. {\em Journal of the American Statistical Association\/}. 116(553): 1--13.
  \item Wang, T., and Yan, J. (2021): Discussion of ``On studying extreme values and systematic risks with nonlinear time series models and tail dependence measures''. {\em Statistical Theory and Related Fields\/}. 5(1): 38--40.
  \item *Wu, J., Chen, M.-H., Schifano, E. D., and Yan, J. (2021): Online updating of survival analysis. {\em Journal of Computational and Graphical Statistics\/}. 30(4): 1209--1223.
  \item *Xue, Y., Yan, J., and Schifano, E. D. (2021): Simultaneous monitoring for regression coefficients and baseline hazard profile in Cox modeling of time-to-event data. {\em Biostatistics\/}. 22(4): 756--771.
  \item *Yuan, Y., Yan, J., Zhang, P. (2021): Assortativity measures for weighted and directed networks. {\em Journal of Complex Networks\/}. 9(2): cnab017. 
  \item *Doshi, R., Yan, J., and Aseltine, R. (2020): Age differences in racial/ethnic disparities in preventable hospitalizations for heart failure in Connecticut, 2009-2015: A population-based longitudinal study. {\em Public Health Reports\/} 135(1): 56--65.
  \item *Hu, C., Pozdnyakov, V., and Yan, J. (2020): Density and distribution evaluation for convolution of independent gamma variables. {\em Computational Statistics\/}. 35(1): 327--342.
  \item *Jiang, Y., He, X., Lee, M-L. T., Rosner, B., and Yan, J. (2020): Rank-based tests for clustered data with R package clusrank. {\em Journal of Statistical Software\/}. 96(6): 1--26.
  \item *Li, Y., Li, Y., Qin, Y., and Yan, J. (2020): Copula modeling for data with ties.  {\em Statistics and Its Interfaces\/}. 13(1): 103--117.
  \item Pozdnyakov, V., Elbroach, L. M., Hu, C., Meyer, T., and Yan, J. (2020): On estimation for Brownian motion governed by telegraph process with multiple off states. {\em Methodology and Computing in Applied Probability\/} 22: 1275--1291.
  \item *Vaughan, G., Aseltine, R., Chen, K., and Yan, J. (2020): Efficient interaction selection for clustered data via stagewise generalized estimating equations. {\em Statistics in Medicine\/}. 39(22): 2855--2868.
  \item *Wang, W., Aseltine, R., Chen, K., and Yan, J. (2020): Integrative survival analysis with uncertain event times in application to a suicide risk study. {\em Annals of Applied Statistics\/}. 14(1): 51-73.      
  \item *Wang, C., Schifano, E. D., and Yan, J. (2020): Geographic ratings with spatial random effects in a two-part model. {\em Variance\/}. 13(1): 141--160.
  \item Xu, G., Chiou, S., Yan, J., Marr, K., and Huang, C.-Y. (2020): Generalized scale-change models for recurrent event processes under informative censoring. {\em Statistica Sinica\/}. 30(4): 1773--1795.
  \item *Xue, Y., Wang, H., Yan, J., and Schifano, E. D. (2020), An online updating approach for testing the proportional hazards assumption with streams survival data. {\em Biometrics\/}. 76(1): 171--182.
  \item Aseltine, R., *Wang, W., Benthien, R., Katz, M., Wagner, C., Yan, J., and Lewis, C. (2019): Reductions in race and ethnic disparities in hospital readmissions following total joint arthroplasty from 2005--2015. {\em Journal of Bone and Joint Surgery\/}. 101(22) 2044--2050. 
 \item Caplan, D. J., *Li, Y., *Wang, W., Kang, S., Marchini, L., Cowen, H. J., and Yan, J. (2019): Dental restoration longevity among geriatric and special needs patients. {\em JDR Clinical \& Translational Research\/}. 4(1): 41--48.
  \item Chiou, S., Huang, C., Xu, G., and Yan, J. (2019): Semiparametric regression analysis of panel count data: A practical review. {\em International Statistical Review\/}. 87(1), 24--43.
  \item Pozdnyakov, V., Elbroch, M., Labarga, J. A., Meyer, T., and Yan, J. (2019): Discretely observed Brownian motion governed by telegraph process: Estimation. {\em Methodology \& Computing in Applied Probability\/} 21(3): 907--920.
  \item *Bader, B., Yan, J., and Zhang, X. (2018): Automated threshold selection for extreme value analysis via goodness-of-fit tests with application to return level mapping. {\em Annals of Applied Statistics\/} 12(1): 310--329.
  \item Chiou, S., Xu, G., Yan, J., and Huang, C. (2018): Semiparametric estimation of the accelerated mean model for panel count data with informative examination times. {\em Biometrics\/}, 74(3): 944--953.
  \item *Wang, C., Chen, M.-H., Wu, J., Yan, J., Zhang, Y., and Schifano, E. D. (2018): Online updating method with new variables for big data streams. {\em Canadian Journal of Statistics\/} 46(1): 123--146.
 \item *Bader, B., Yan, J., and Zhang, X. (2017): Automated selection of $r$ for the $r$ largest order statistics approach with adjustment for sequential testing. {\em Statistics and Computing\/} 27(6): 1435--1451.
  \item *Vaughan, G., Aseltine, R.~H, Chen, K., and Yan, J. (2017): Stagewise estimating equations with grouped variables. {\em Biometrics\/} 73(4): 1332--1342.
  \item *Wang, Z., Jiang, Y., Wan, H., Yan, J., and Zhang, X. (2017): Detection and attribution of changes in extreme temperatures at regional level.  {\em Journal of Climate\/} 30(17): 7035--7047.
  \item Xu, G., Chiou, S., Huang, C.-Y., Wang, M.-C., and Yan, J. (2017): Joint scale-change models for recurrent events and failure time. {\em Journal of the American Statistical Association\/} 112: 794--805.
  \item Olayivola, J. N., Adnerson, D. R., Jepeal, N., Aseltine, R. H., Pickett, C., Yan, J., and Zlateva, I. (2016): Electronic consultations to improve the primary care-specialty care interface for cardiology in the medically underserved: A cluster-randomized controlled trial. {\em Annals of Family Medicine\/} 14(2): 133--140.
  \item Schifano, E. D., Wu, J., Wang, C., Yan, J. and Chen, M.-H. (2016): Online updating of statistical inference in the big data setting. {\em Technometrics\/} 58(3): 393--403.
  \item *Wang, C., Chen, M.-H., Schifano, E. D., Wu, J., and Yan, J. (2016): Statistical methods and computing for big data. {\em Statistics and Its Interfaces\/} 9(4): 399--414.
  \item *Wang, W., Chen, M.-H., Chiou, S., Lai, H.-C., Wang, X., Yan, J., and Zhang, Z. (2016): Onset of persistent Pseudomonas Aeruginosa infection in children with cystic fibrosis with interval censored data. {\em BMC Medical Research Methodology\/} 16(122): 1--10.
  \item Aseltine, R.~H., Yan, J., Fleischman, S., Katz, M., and DeFrancesco M. (2015): Race and ethnic disparities in hospital readmissions following vaginal and cesarean delivery. {\em Obstetrics \& Gynecology\/} 126(5): 1040--1047.
  \item Aseltine, R.~H., Yan, J., Gruss, C.~B., Wagner, C., and Katz, M. (2015): Connecticut hospital readmissions related to chest pain and heart failure: {D}ifferences by race, ethnic, and payer. {\em Connecticut Medicine\/} 79(2):  69--76.
  \item Chi, Z., Pozdnyakov, V., and Yan, J. (2015): On occupation time of Brownian motion. {\em Statistics and Probability Letters\/} 97: 83--87.
  \item *Chiou, S., Kang, S., and Yan, J. (2015): Semiparametric accelerated failure time modeling for clustered failure times from stratified sampling. {\em Journal of the American Statistical Association\/} 110: 621--629.
  \item *Chiou, S., Kang, S., and Yan, J. (2015): Rank-based estimating equations with general weight for accelerated failure time models: An induced smoothing approach. {\em Statistics in Medicine\/} 34(9): 1495--1510.
  \item Kojadinovic, I., Shang, H., and Yan, J. (2015): A class of goodness-of-fit tests for spatial extremes models based on max-stable processes. {\em Statistics and Its Interfaces\/} 8(1): 45--62.
  \item *Prates, M.~O., Dey, D.~K., Willig, M.~R., and Yan, J. (2015): Transformed Gaussian Markov random fields and spatial modeling. {\em Spatial Statistics\/} 14(C), 382--399.
  \item *Shang, H., Yan, J., and Zhang, X. (2015): A two-step approach to model precipitation extremes in {C}alifornia based on max-stable and marginal point processes. {\em The Annals of Applied Statistics\/} 9(1): 452--473.
  \item *Chiou, S., Kang, S., Kim, J., and Yan, J. (2014): Marginal semiparametric multivariate accelerated failure time model with generalized estimating equations. {\em Lifetime Data Analysis\/} 20(4): 599--618.
  \item *Chiou, S., Kang, S., and Yan, J. (2014): Fitting accelerated failure time models in routine survival analysis. {\em Journal of Statistical Software\/} 61(11): 1--23.
  \item *Chiou, S., Kang, S., and Yan, J. (2014): Fast accelerated failure time modeling for case-cohort data. {\em Statistics and Computing\/} 24(4): 559--568.
  \item Pozdnyakov, V., Meyer, T., Wang, Y., and Yan, J. (2014): On modeling animal movement using Brownian motion with measurement error. {\em Ecology\/} 95(2): 247--253.
  \item *Wang, Z., Yan, J., and Zhang, X. (2014): Incorporating spatial dependence in regional frequency analysis. {\em Water Resources Research\/} 50(12): 9570--9585.
  \item Yan, J., Chen, Y., Lawrence-Apfel, K., Ortega, I. M., Pozdnyakov, V., Williams, S., and Meyer, T. (2014): A moving-resting process with an embedded Brownian motion for animal movements. {\em Population Ecology\/} 56(2): 401--415.
  \item Yan, J., Guo, C., Paarlberg, L.~E. (2014): Are antipoverty nonprofit organizations located where they are needed? Spatial analysis of the Greater Hartford region. {\em American Statistician\/} 68(4): 243--252.
  \item *Prates, M.~O., Aseltine, R.~H., Dey, D.~K., and Yan, J. (2013): Assessing intervention efficacy on high risk drinkers using generalized linear mixed models with a new class of link functions. {\em Biometrical Journal\/} 55(6): 912--924.
  \item *Wang, X., Chen, M.-H., and Yan, J. (2013): Bayesian dynamic regression model for interval censored data. {\em Lifetime Data Analysis\/} 19(3): 297--316.
  \item *Wang, X., Ma, S., and Yan, J. (2013): Augmented estimating equations for semiparametric panel count regression with informative observation times and censoring time. {\em Statistica Sinica\/} 23(1): 359--381.
  \item *Wang, X. and Yan, J. (2013): Practical notes on multivariate modeling based on elliptical copulas. {\em Journal of the French Statistical Society\/} 154(1): 102--115.
  \item Yan, J., Aseltine, R., and Harel, O. (2013): Comparing regression coefficients between nested models for clustered data with generalized estimating equations. {\em Journal of Educational and Behavioral Statistics\/} 38(2): 172--189.
  \item Cavallo, A., Rosenthal, B., Wang, X., and Yan, J. (2012): Treatment of the data collection threshold in operational risk: {A} case study using the lognormal distribution. {\em Journal of Operational Risk\/} 7(1): 3--38.
  \item Chen, D. C.~R., Kirshenbaum, D.~S., Yan, J., Kirshenbaum, E., and Aseltine,  R.~H. (2012): Characterizing changes in student empathy throughout medical school. {\em Medical Teacher\/} 34(4): 305--–311.
  \item Havens, E.~K., Martin, K.~S., Yan, J., Dauser-Forrest, D., and Ferris, A.~M. (2012): Federal nutrition program changes and healthy food availability. {\em  American Journal of Preventive Medicine\/} 43(4): 419--422.
  \item Kojadinovic, I. and Yan, J. (2012): A nonparametric test of exchangeability for bivariate extreme-value copulas. {\em Scandinavian Journal of Statistics\/} 39(3): 480--496.
  \item Kojadinovic, I. and Yan, J. (2012): Goodness-of-fit testing based on a weighted bootstrap: A fast large-sample alternative to the parametric bootstrap. {\em Canadian Journal of Statistics\/} 40(3): 480--500.
  \item Yan, J. and Huang, J. (2012): Model selection for time-varying coefficient Cox models. {\em Biometrics\/} 68(2): 419--428.
  \item Yan, J., Liao, G.-Y., Gebremichael, M., Shedd, R., and Vallee, D. (2012): Characterizing the Uncertainty in River Stage Forecasts Conditional on Point Forecast Values. {\em Water Resources Research\/} 48: W12509.
  \item Allignol, A., Latouche, A., Yan, J., and Fine, J.~P. (2011):  A regression model for the conditional probability of a competing event: {A}pplication to monoclonal gammopathy of unknown significance. {\em Journal of Royal Statistical Society, Series C: Applied Statistics\/}  60(1): 135--142.
  \item Gebremichael, M., Liao, G.-Y., and Yan, J. (2011): Non-parametric error model for high resolution satellite rainfall products. {\em Water Resources Research\/} 47: W07504--W07512.
  \item Genest C., Kojadinovic I., Ne\v{s}lehov\'a J., and Yan J. (2011): A goodness-of-fit test for bivariate extreme value copulas. {\em Bernoulli\/} 17(1): 253-275.
  \item Guan, Y., Yan, J., and Sinha, R. (2011): Variance estimation for statistics computed from single recurrent event processes. {\em Biometrics\/} 17(3): 711--718.
  \item Harel, O., Mukhopadhyay, N., and Yan, J. (2011): On a sequential probability ratio rest subject to incomplete data. {\em Sequential Analysis\/} 30: 441--456.
  \item Kojadinovic, I., Segers, J., and Yan, J. (2011): Large-sample tests of extreme-value dependence for multivariate copulas. {\em Canadian Journal of Statistics\/} 39(4): 703--720.
  \item Kojadinovic, I. and Yan, J. (2011): Tests of serial independence for multivariate time series based on a {M\"obius} decomposition of the independence empirical copula process. {\em Annals of the Institute of Statistical Mathematics\/} 63(2): 347--373.
  \item Kojadinovic, I. and Yan, J. (2011): A goodness-of-fit test for multivariate multiparameter copulas based on multiplier central limit theorems. {\em Statistics and Computing\/} 21(1): 17--30.
  \item Kojadinovic, I, Yan, J., and Holmes, M. (2011): Fast large-sample goodness-of-fit for copulas. {\em Statistica Sinica\/} 21(2): 841--871.
  \item *Prates, M.~O., Dey, D.~K, Willig, M.~R. and Yan, J. (2011): Intervention analysis of the hurricane impact on snail abundance in a tropical forest with spatio-temporal data.  {\em Journal of Agricultural, Biological, and Ecological Statistics\/} 16(1): 142--156.
  \item *Shang, H., Yan, J., Gebremichael, M., and Ayalew, S.~M. (2011): Trend analysis of extreme precipitation in the northwestern highlands of Ethiopia with a case study of Debre Markos. {\em Hydrology and Earth System Sciences\/} 15(3): 1937--1944.
  \item *Shang, H., Yan, J., and Zhang, X. (2011): ENSO influence on winter maximum daily precipitation in California in a spatial model. {\em Water Resources Research\/} 47: W11507--W11515.
  \item *Wang, X. and Yan, J. (2011): Fitting semiparametric regressions for panel count survival data with an {R} package {spef}. {\em Computer Methods and Programs in Biomedicine\/} 104(2): 278--285.
  \item Kojadinovic, I. and Yan, J. (2010): Nonparametric rank-based tests of bivariate extreme-value dependence. {\em Journal of Multivariate Analysis\/} 101(9): 2234--2249.
  \item Kojadinovic, I. and Yan, J. (2010): Comparison of three semiparametric methods for estimating dependence parameters in copula models. {\em Insurance: Mathematics and Economics\/} 47(1): 52--63.
  \item Kojadinovic, I. and Yan, J. (2010): Modeling multivariate distributions with continuous margins using the copula R package. {\em Journal of Statistical Software\/} 34(9): 1--20.
  \item *Wang, X., Gebremichael, M., and Yan, J. (2010): Weighted likelihood copula modeling of extreme rainfall events in Connecticut. {\em Journal of Hydrology\/} 390(1--2): 108--115.
  \item Yan, J. and the Academic ED SBIRT Research Collaborative. (2010): The impact of screening, brief intervention and referral for treatment in emergency department patients' alcohol use: A 3-, 6- and 12-month follow-up. {\em Alcohol \& Alcoholism\/} 45(6): 514--519.
  \item Yan, J., Cheng, Y., Fine, J.~P., and Lai, H.-C. (2010): Uncovering symptom progression history from disease registry data with application to young cystic fibrosis patients. {\em Biometrics\/} 66(2): 594--602.
  \item Cowles, M.~K., Yan, J., and Smith, B.~J. (2009): Reparameterized and marginalized posterior and predictive sampling for complex Bayesian geostatistical models. {\em Journal of Computational and Graphical Statistics\/} 18(2): 262--282.
  \item Yan, J. and Gebremichael, M. (2009): Estimating actual rainfall from satellite rainfall products. {\em Atmospheric Research\/} 92(4): 481--488.
  \item Yan, J. and Huang, J. (2009): Partly functional temporal process regression with semiparametric profile estimating functions. {\em Biometrics\/} 65(2): 431--440.
  \item Smith, B.~J., Yan, J., and Cowles, M.~K. (2008): Unified geostatistical modeling for data fusion and spatial heteroskedasticity with R package ramps. {\em Journal of Statistical Software\/} 25(10): 1--21.
  \item Yan, J. and Fine, J.~P. (2008): Analysis of episodic data with application to recurrent pulmonary exacerbations in cystic fibrosis patients. {\em Journal of the American Statistical Association\/}  103: 498--510.
  \item Stramer, O. and Yan, J. (2007): Asymptotics of an efficient {M}onte {C}arlo estimation for the transition density of diffusion processes. {\em Methodology \& Computing in Applied Probability\/} 9(4): 483--496.
  \item Stramer, O. and Yan, J. (2007): On simulated likelihood of discretely observed diffusion processes and comparison to closed-form approximation. {\em Journal of Computational and Graphical Statistics\/} 16(3): 672--691.
  \item Yan, J. (2007): Enjoy the joy of copulas. {\em Journal of Statistical Software\/} 21(4): 1--21.
  \item Yan, J. (2007): Spatial stochastic volatility for lattice Data. \emph{Journal of Agricultural, Biological, and Environmental Statistics} 12(1): 25--40.
  \item Yan, J., Cowles, M.~K., Wang, S., and  Armstrong, M.~P. (2007): Parallelizing MCMC for Bayesian spatiotemporal geostatistical models. {\em Statistics and Computing\/} 17(4): 323--335.
  \item Yan, J. and Tamboli, C.~P. (2007): Testing concordance of clinical characteristics in familial studies with application to inflammatory bowel diseases. {\em Biometrical Journal\/} 49(6): 840--853.
  \item Halekoh, U., H\o{}jsgaard, S., and Yan, J. (2006): The {R} package geepack for generalized estimating equations. {\em Journal of Statistical Software\/} 15(2): 1--11.
  \item Yan, J. and Fine, J.~P. (2005): Functional association models for multivariate survival processes. {\em Journal of the American Statistical Association\/} 100(469): 184--196.
  \item Fine, J.~P., Yan, J., and Kosorok, M.~R. (2004): Temporal process regression. {\em Biometrika\/} 91(3): 683--703.
  \item Yan, J. and Fine, J.~P. (2004): Estimating equations for association structures ({P}kg: P859-880). {\em Statistics in Medicine\/} 23(6): 859--874.
  \item Yan, J. and Fine, J.~P. (2004): Reply to comment on ``Estimating equations for association structures'' ({P}kg: 859--880). {\em Statistics in Medicine\/} 23(6): 879--880.
  \end{RefereedJournalArticles}
  \begin{UnderReview}
  \item Li, Yan, Wang, T., Yan, J., and Zhang, X. (2024): Improved optimal fingerprinting based on estimating equations reaffirms anthropogenic effect in global warming.
  \item *Chandy, M., Schifano, E. D., Yan, J., and Zhang, T. (2024): Nonparametric block bootstrap Kolmogorov--Smirnov goodness-of-fit test.
  \item *Godoy, L., Prates, M., and Yan, J. (2024): Statistical inferences and predictions for areal data and spatial data fusion with Hausdorff-–Gaussian processes.
  \item *Louit, S., Clark, E., Gelbard, Al, Vivek, N., Yan, J., and Zhang, P. (2024): CALF-SBM: A covariate-assisted latent factor stochastic block model.
  \item Shen, O., Feng, Q., Yan, J., and Zhang, P. (2024): Rank-based assortativity for weighted, directed networks.
  \item *Hughes, S., Matthews, G., and Yan, J. (2024): Statistical evaluation of outdoor field hockey penalty corners.
  \item *Fiore, O., Schifano, E. D., and Yan, J. (2023): On Devon Allen’s disqualification at the 2022 World Track and Field Championships.
  \item *Wang, T., *Xiao, S., and Yan, J. (2023): Comparison of sectoral structures between China and Japan: A network perspective.
  \item Chiou, S., Aseltine, R., Schilling, E., Lutz, K., and Yan, J. (2022): A bivariate two-part model for censored durations of depression and relational stressor in young adults.
  \item *Godoy, L., Prates, M., and Yan, J. (2022): Model-based Voronoi linkage between point-referenced data and areal data in spatial analysis with application to Brazilian election 2018.
  \item *Ma, S., Yan, J., and Zhang, X. (2020): Extreme value modeling with generalized Pareto distributions for rounded data.
%  \item Yu, B. and Yan, J. (2016): Estimating relative risks for longitudinal binary response data. % Under revision for \emph{Statistics in Medicine}.
  % \item Larose, C. D., Harel, O., and Yan, J. (2014): Estimating regression coefficient change in nested models with incomplete data.  % Under review at \emph{J. App. Stat}.
  \end{UnderReview}
  \begin{Software}
  \item *Bader, B. and Yan, J.: R package \texttt{eva} on CRAN, extreme value analysis.
  \item *Chiou, S., Kang, S., and Yan, J.: R package \texttt{aftgee} on CRAN, multivariate accelerated failure time modeling with generalized estimating equations.
  \item *Chiou, S., Wang, X. and Yan, J: R package \texttt{spef} on CRAN, semiparametric estimating functions.
  \item Hofert, M., Kojadinovic, I.,  M\"achler, M. and Yan, J.: R package \texttt{copula} on CRAN, multivariate dependence with copula.
  \item *Hu, C., Pozdnyakov, V., and Yan, J.: R package \texttt{coga} on CRAN, convolution of gamma variables.
  \item *Hu, C., Yan, J., and Pozdnyakov, V.: R package \texttt{smam} on CRAN, statistical modeling of animal movement. 
  \item *Jiang, Y., Lee, M.-L. T., and Yan, J.: R package \texttt{clusrank} on CRAN, rank-based tests for clustered data.
  \item Kojadinovic, I. and Yan, J.: R package \texttt{fgof} on CRAN, fast goodness-of-fit test.
  \item *Li, Y., Chen, K., and Yan, J.: R package \texttt{tls} on CRAN, total least squares.
  \item *Li, Y., Chen, K., and Yan, J.: R package \texttt{dacc} on CRAN, detection and attribution of climate change.
  \item *Li, Y., Wang, W., and Yan, J.: R package \texttt{touch} on CRAN, tools of utilization and cost in healthcare.
  \item *Prates, M. O., Wang, W., and Yan, J.: R package \texttt{rbugs} on CRAN, fusing R with OpenBugs.
  \item Smith, B. P., Yan, J., and Cowles, M. K.: R package \texttt{ramps} on CRAN, reparametrized and marginalized posterior sampling.
  \item *Vaughan, G., Chen, K., and Yan, J.: R package \texttt{sgee} on CRAN, stagewise generalized estimating equations.
  \item *Wang, W., Chen, K., and Yan, J.: R package \texttt{intsurv} on CRAN, integrative survival analysis.
  \item *Wang, W., Fu, H., and Yan, J.: R package \texttt{reda} on CRAN, recurrent event data analysis.
  \item *Wang, W. and Yan, J.: R package \texttt{splines2} on CRAN, regression spline functions and classes too.
  \item *Wang, X., Chen, M.-H., and Yan, J.: R package \texttt{dynsurv} on CRAN, dynamic survival modeling.
  \item *Xiao, S., Yan, J., and Zhang, P.: R package \texttt{fcstat} on GitHub, statistical methods for estimating functional connectivity analysis in brain networks.
  \item *Xiao, S., Yan, J., and Zhang, P.: R package \texttt{ionet} on CRAN, input-output networks.
  \item Yan, J.: R package \texttt{tpr} on CRAN, temporal process regression.
  \item Yan, J.: R package \texttt{som} on CRAN, self-organizing map with application to gene clustering.
  % \item Yan, J.: R package \texttt{species} to be on CRAN, estimating the number of species.
  \item Yan, J., H\o{}jsgaard, S., and Halekoh, U.: R package \texttt{geepack} on CRAN, generalized estimating equation package.
  \item Yan, J.: R package \texttt{KMsurv} on CRAN,  datasets and functions for Klein and Moeschberger (1997), ``Survival Analysis, Techniques for Censored and Truncated Data'', Springer.
  \item *Yuan, Y., Wang, T., Yan, J., and Zhang, P.: R package \texttt{wdnet} on CRAN, weighted directed networks.
  \end{Software}
  \begin{NonRefereedPublications}
  \item Yan, J. (2020): A reformed Journal of Data Science for the era of data science. {\em Journal of Data Science\/}, 18(3): 405--406.
  \item Follman, D., Song, P. X.-K., Wang, H., and Yan, J. (2020): Data science in action in response to the outbreak of COVID-19. {\em Journal of Data Science\/} 18(3): 407--408.
  \item Yan, J. (2004): Fusing {R} and {BUGS} through {Wine}. {\em R News\/} 4(2): 19--21.
  \item Yan, J. and Rossini, A. (2003): Building {M}icrosoft {W}indows versions of {R} and {R} packages under {I}ntel {L}inux. {\em R News\/} 3(1): 15--17.
  \item Yan, J. (2002): geepack: Yet another package for generalized estimating equations. {\em R News\/} 2(3): 12--14.
  \end{NonRefereedPublications}
  \begin{BookReviews}
  \item Yan, J. (2006): Gaussian Markov random fields: Theory and applications. {H}arvard {R}ue and {L}eonhard {H}eld. {\em Journal of the American Statistical Association\/} 101(473): 388--389.
  \item Yan, J. (2005): Analysis of multivariate survival data. {P}hillip {H}ougaard. {\em Journal of the American Statistical Association\/} 100(469): 355--356.
  \item Yan, J. (2004): Bayesian survival analysis. {J}oseph {G}. {I}brahim, {M}ing-{H}ui {C}hen, and {D}ebajyoti {S}inha. {\em Journal of the American Statistical Association\/} 99(468): 1202--1203.
  \item Yan, J. (2004): Survival analysis: Techniques for censored and truncated data (2nd ed.). {J}ohn {P}. {K}lein and {M}elvin {L}. {M}oeschberger. {\em Journal of the American Statistical Association\/} 99(467): 900--901.
  \end{BookReviews}
  \end{Publications}
  % \begin{Awards}
  % \item Innovations in Instructional Computing Award --- Academic Technologies Advisory Council, University of Iowa, 2007. PI: Mary Kathryn Cowles, Co-PIs: Marc P. Armstrong, Brian J. Smith, Shaowen Wang, and Jun Yan.
  % \end{Awards}
  \begin{InvitedTalksLectures}
  \begin{InvitedTalks}
  \item Recurrent Events Modeling Based on a Reflected Brownian Motion with Application to Hypoglycemia, 05/23/2024, New England Statistics Symposium 2024.
  \item Recurrent Events Modeling Based on a Reflected Brownian Motion with Application to Hypoglycemia, 04/18/2024, Department of Biostatistics, University of Pittsburgh.
  \item Optimal Fingerprinting in Climate Change Detection and Attribution with Estimating Equations, 11/20/2023, Department of Statistics, Oregon State University (virtual).
  \item Introduction to Survival Analysis, 10/17/2023, Department of Statistics, University of Lagos, Nigeria (Virtual).
  \item Recurrent Events Modeling Based on a Reflected Brownian Motion with Application to Hypoglycemia, 03/30/2023, Department of Statistics, Kansas State University (virtual).
  \item Optimal Fingerprinting with Estimating-Equations, 08/22/2022, Learning the Earth with Artificial Intelligence and Physics Center, Columbia University.
  \item Introductory Overview Lecture: Sports Analytics Beyond Performance Evaluation, 08/08/2022, Joint Statistical Meetings.
  \item Optimal Fingerprinting with Bias-Corrected Estimating Equations, 06/11/2022, Special Conference Celebrating the 80th Anniversary of Renmin, Institute of Statistics and Big Data, Renmin University of China (virtual).
  \item Recurrent Events Modeling Based on a Reflected Brownian Motion with Application to Hypoglycemia, 05/18/2022, Department of Applied Mathematics, The Hong Kong Polytechnic University (virtual).
  \item Recurrent Events Modeling Based on a Reflected Brownian Motion with Application to Hypoglycemia, 03/29/2022, ENAR Spring Meeting (virtual).
  \item Optimal Fingerprinting with Bias-Corrected Estimating Equations, 01/31/2022, International Detection and Attribution Group (virtual).
  \item Brownian Motion Governed by Telegraph Process in Modeling High-Frequency Financial Series, 08/27/2021, Department of Statistics, Universidade Federal de Minas Gerais (virtual).
  \item Correct Working Correlation for Generalized Estimating Equations May Lead to More Bias Under Measurement Error than Working Independence, 06/09/2021, Department of Applied Statistics, Yonsei University (virtual).
  \item Moving-Resting Process with Measurement Error in Animal Movement Modeling, 10/26/2020, Center for Statistical Science, Tsinghua University (virtual).
  \item An Applied Statistician's Adventure: Climate Change, Animal Movement, Sports Analytics, and Beyond, 05/09/2020, Clubear Lecture (virtual).
  \item An Online Updating Approach for Testing the Proportional Hazards Assumption with Streams of Survival Data, 11/29/2019, The 6th Workshop on Survival Analysis and Applications, University of San Paulo, Brazil.
  \item Integrative Survival Analysis with Uncertain Event Times in Application to a Suicide Risk Study, 07/05/2019, School of Statistics, Shanxi University of Finance and Economics.
  \item Acrobatic Regression in Detection and Attribution of Climate Change, 07/03/2019, Guanghua School of Management, Peking University.
  \item An Online Updating Approach for Testing the Proportional Hazards Assumption with Streams of Survival Data, 06/30/2019, School of Mathematics, Jilin University
  \item Fingerprinting Changes in Climate Extremes with Joint Modeling of Observations and Climate Model Simulation, 07/30/2018, Vancouver, BC, Canada, Joint Statistical Meetings.
  \item Generalized Scale-Change Models for Recurrent Event Processes under Informative Censoring, 07/03/2018, 2018 ICSA China Conference with the Focus on Data Science, Qingdao, China.
  \item Growth Curve Analysis with Shape-restricted Splines, 07/01/2018, International Statistics Forum, Renmin University of China, Beijing, China.
  \item Generalized Scale-Change Models for Recurrent Event Processes under Informative Censoring, 06/30/2018, School of Mathematics, Jilin University.
  \item Optimal Fingerprinting in Detection and Attribution of Changes in Climate Extremes, 06/26/2018, Center for Statistical Science, Peking University.
  \item Statistical Methods for Big Stream Data, 06/19/2018, Shanxi University of Finance and Economics
  \item Things about Being a Professor, 12/20/2017, School of Statistics, Renmin University of China.
  \item Online Updating Method with New Variables for Big Data Streams, 12/17/2017, School of Statistics, Shanxi University of Finance and Economics.
  \item Balancing the Bias-Variance Tradeoff in Extreme Value Analysis, 08/31/2017, Center for Mathematical Research, University of Montr\'eal.
  \item Stagewise Generalized Estimating Equations with Grouped Variables, 03/02/2017, Department of Mathematics and Statistics, Boston University.
  \item Online Updating Method with New Variables for Big Data Streams, 10/12/2016, Department of Statistics and Biostatistics, Rutgers University.
  \item Semiparametric Accelerated Failure Time Modeling for Clustered Failure Times From Stratified Sampling, 07/02/2016, IBS-China 4th International Biostatistics Symposium, Shanghai, China.
  \item Spatial estimating equations for detection and attribution of changes in climate extremes, 05/29/2016, China R Conference, Beijing, China.
  \item Spatial estimating equations for detection and attribution of changes in climate extremes, 02/02/2016, International Detection and Attribution Group Meeting, Boulder, CO\@.
  \item Spatial estimating equations with application to changes in climate extremes, 10/26/2015, Department of Mathematical Sciences, Worcester Polytechnic Institute.
  \item Optimal fingerprinting in detection and attribution of changes in climate extremes with combined score equations, 06/20/2015, Alumni Symposium, School of Statistics, Renmin University of China, Beijing, China.
  \item Onset time of chronic pseudomonas aeruginosa infection of cystic fibrosis patients with interval censored data, 06/16/2015, Applied Statistics Symposium, International Chinese Statistical Association, Fort Collins, CO\@.
  \item A bivariate two-part model to assess the effect of coping strategy on stressor and depression, 06/06/2015, Frontiers in Applied and Computational Mathematics 2015, New Jersey Institute of Technology (NJIT) in Newark, New Jersey.
  \item Incorporating spatial dependence in regional frequency analysis, 05/25/2014, International Statistics Forum, Renmin University of China, Beijing China.
  \item A partial review of software for big data statistics, 02/12/2014, Statistical and Computational Theory and Methodology for Big Data Analysis, Banff International Research Station, Banff, Alberta, Canada.
  \item Statistics methods and computing for semiparametric accelerated failure time models with induced smoothing, 05/13/2013, Department of Biostatistics, Brown University.
  \item Transformed Gaussian Markov random fields and Spatial Modeling, 11/23/2012, Universitdade Federal de Minas Gerais, Brazil.
  \item Transformed Gaussian Markov random fields and Spatial Modeling, 10/05/2012, Department of Biostatistics, University of Massachusetts---Amherst.
  \item Fast accelerated failure time model for case-cohort data, 06/24/2012, Boston, MA, International Chinese Statistical Association Applied Statistics Symposium.
  \item Model selection for Cox models with time-varying coefficients, 04/21/2012, Boston University, New England Statistics Symposium.
  \item Max-Stable processes for spatial extremes modeling: A review and some ongoing research, 03/01/2012, Climate Research Division, Environmental Canada.
  \item Multivariate accelerated failure time models with generalized estimating equations, 10/04/2011, Biostatistics Research Branch, National Institute of Allergy and Infectious Disease.
  \item Augmented estimating equations for semiparametric panel count regression with informative censoring, 03/23/2011, ENAR Spring Meeting.
  \item Nonparametric rank-based tests of bivariate extreme-value dependence, 10/27/2010, UMass-UConn Joint Statistics Symposium.
  \item Augmented estimating equations for semiparametric panel count regression with informative censoring, 06/30/2010, Yunnan University, International Conference on Statistical Analysis of Complex Data.
  \item Nonparametric rank-based tests of bivariate extreme-value dependence, 04/17/2010, Harvard University, 2010 New England Statistics Symposium.
  \item Combining data for efficient prediction of the spatial distribution of Iowa residential radon levels, 08/02/2009, Washington, DC, Joint Statistical Meetings.
  \item Fast large sample goodness-of-fit test for copulas, 10/02/2008, D\'epartement de math\'ematiques et de statistique, Universit\'e Laval.
  \item Tests of serial independence for multivariate time series based on a M\"obius decomposition of the independence empirical copula process, 06/21/2008, Renmin University of China, International Statistics Forum 2008.
  \item Partly functional temporal process regression with semiparametric profile estimating functions, 01/29/2008, Division of Biostatistics, Yale University.
  \item Spatial stochastic volatility, 07/12/2006, Beijing, China, Far Eastern Meeting of the Econometrics Society (FEMES) 2006.
  \item Partly functional temporal process regression, 06/28/2006, Hong Kong, China, INFORMS International Conference 2006.
  \item Spatial stochastic volatility, 04/21/2006, Department of Economics, University of Illinois -- Urbana-Champaign.
  \item Temporal process regression, 08/10/2004, Toronto, Canada, Joint Statistical Meeting.
  \end{InvitedTalks}
  \begin{InvitedWorkshops}
  \item June 2024, Climate Change Detection and Attribution with Estimating Equations (2-hour short course with Yan Li), 15th International Meeting on Statistical Climatology, Toulouse, France.
  \item June 2023, Applied Event Time Data Analysis with R (1-day course with Steven Chiou). 2023 ICSA Applied Statistics Symposium, Ann Arbor, Michigan.
  \item May 2022, Applied Event Time Data Analysis with R (1-day course with Steven Chiou). 2022 New England Statistics Symposium, Storrs, CT.
  \item July 2016, Advanced Statistical Computing (20-hour course). Shanghai University of Finance and Economics, Shanghai, China.
  \item June 2015, Advanced Statistical Computing (20-hour workshop). Shanghai University of Finance and Economics, Shanghai, China.
  \item April 2015, Modern Multivariate Statistical Learning: Methods and Applications (1-day short course with Kun Chen). The 29th NESS at University of Connecticut, Storrs, CT.
  \item May 2014, Advanced Statistical Computing (32-hour course). Renmin University of China, Beijing China.
  \item April 2013, Statistical Analysis of Spatial Data and Visualization with Google Map (1-day short course with Marcos Prates). The 27th NESS at University of Connecticut, Storrs, CT.
  \item November 2012, Introduction to the Theory and Practice of Copulas (1-week short course). Universidade Federal de Minas Gerais, Belo Horizonte, Minas Gerais, Brazil.
  \item April 2011, Introduction to the Theory and Practice of Copulas (1-day short course with Ivan Kojadinovic). The 25th NESS at University of Connecticut. Storrs, CT.
  \item May 2008, Introduction to the Theory and Practice of Copulas (32-hour course). Renmin University of China, Beijing, China.
  \end{InvitedWorkshops}
\end{InvitedTalksLectures}
\end{vita}
\end{document}
